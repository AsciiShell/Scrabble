%!TEX TS-program = xelatex

% Шаблон документа LaTeX создан в 2018 году
% Алексеем Подчезерцевым
% В качестве исходных использованы шаблоны
% 	Данилом Фёдоровых (danil@fedorovykh.ru) 
%		https://www.writelatex.com/coursera/latex/5.2.2
%	LaTeX-шаблон для русской кандидатской диссертации и её автореферата.
%		https://github.com/AndreyAkinshin/Russian-Phd-LaTeX-Dissertation-Template

\documentclass[a4paper,14pt]{article}


%%% Работа с русским языком
\usepackage[english,russian]{babel}   %% загружает пакет многоязыковой вёрстки
\usepackage{fontspec}      %% подготавливает загрузку шрифтов Open Type, True Type и др.
\defaultfontfeatures{Ligatures={TeX},Renderer=Basic}  %% свойства шрифтов по умолчанию
\setmainfont[Ligatures={TeX,Historic}]{Times New Roman} %% задаёт основной шрифт документа
\setsansfont{Comic Sans MS}                    %% задаёт шрифт без засечек
\setmonofont{Courier New}
\usepackage{indentfirst}
\frenchspacing

\renewcommand{\epsilon}{\ensuremath{\varepsilon}}
\renewcommand{\phi}{\ensuremath{\varphi}}
\renewcommand{\kappa}{\ensuremath{\varkappa}}
\renewcommand{\le}{\ensuremath{\leqslant}}
\renewcommand{\leq}{\ensuremath{\leqslant}}
\renewcommand{\ge}{\ensuremath{\geqslant}}
\renewcommand{\geq}{\ensuremath{\geqslant}}
\renewcommand{\emptyset}{\varnothing}

%%% Дополнительная работа с математикой
\usepackage{amsmath,amsfonts,amssymb,amsthm,mathtools} % AMS
\usepackage{icomma} % "Умная" запятая: $0,2$ --- число, $0, 2$ --- перечисление

%% Номера формул
%\mathtoolsset{showonlyrefs=true} % Показывать номера только у тех формул, на которые есть \eqref{} в тексте.
%\usepackage{leqno} % Нумерация формул слева	

%% Перенос знаков в формулах (по Львовскому)
\newcommand*{\hm}[1]{#1\nobreak\discretionary{}
	{\hbox{$\mathsurround=0pt #1$}}{}}

%%% Работа с картинками
\usepackage{graphicx}  % Для вставки рисунков
\graphicspath{{images/}}  % папки с картинками
\setlength\fboxsep{3pt} % Отступ рамки \fbox{} от рисунка
\setlength\fboxrule{1pt} % Толщина линий рамки \fbox{}
\usepackage{wrapfig} % Обтекание рисунков текстом

%%% Работа с таблицами
\usepackage{array,tabularx,tabulary,booktabs} % Дополнительная работа с таблицами
\usepackage{longtable}  % Длинные таблицы
\usepackage{multirow} % Слияние строк в таблице
\usepackage{float}% http://ctan.org/pkg/float

%%% Программирование
\usepackage{etoolbox} % логические операторы


%%% Страница
\usepackage{extsizes} % Возможность сделать 14-й шрифт
\usepackage{geometry} % Простой способ задавать поля
\geometry{top=20mm}
\geometry{bottom=20mm}
\geometry{left=20mm}
\geometry{right=10mm}
%
%\usepackage{fancyhdr} % Колонтитулы
% 	\pagestyle{fancy}
%\renewcommand{\headrulewidth}{0pt}  % Толщина линейки, отчеркивающей верхний колонтитул
% 	\lfoot{Нижний левый}
% 	\rfoot{Нижний правый}
% 	\rhead{Верхний правый}
% 	\chead{Верхний в центре}
% 	\lhead{Верхний левый}
%	\cfoot{Нижний в центре} % По умолчанию здесь номер страницы

\usepackage{setspace} % Интерлиньяж
\onehalfspacing % Интерлиньяж 1.5
%\doublespacing % Интерлиньяж 2
%\singlespacing % Интерлиньяж 1

\usepackage{lastpage} % Узнать, сколько всего страниц в документе.

\usepackage{soul} % Модификаторы начертания

\usepackage{hyperref}
\usepackage[usenames,dvipsnames,svgnames,table,rgb]{xcolor}
\hypersetup{				% Гиперссылки
	unicode=true,           % русские буквы в раздела PDF
	pdftitle={Заголовок},   % Заголовок
	pdfauthor={Автор},      % Автор
	pdfsubject={Тема},      % Тема
	pdfcreator={Создатель}, % Создатель
	pdfproducer={Производитель}, % Производитель
	pdfkeywords={keyword1} {key2} {key3}, % Ключевые слова
	colorlinks=true,       	% false: ссылки в рамках; true: цветные ссылки
	linkcolor=black,          % внутренние ссылки
	citecolor=black,        % на библиографию
	filecolor=magenta,      % на файлы
	urlcolor=cyan           % на URL
}

\usepackage{csquotes} % Еще инструменты для ссылок

%\usepackage[style=authoryear,maxcitenames=2,backend=biber,sorting=nty]{biblatex}

\usepackage{multicol} % Несколько колонок

\usepackage{tikz} % Работа с графикой
\usepackage{pgfplots}
\usepackage{pgfplotstable}
\usepackage{algorithm}
\usepackage{algpseudocode}
 
% Добавляем свои блоки
 
\makeatletter
\algblock[ALGORITHMBLOCK]{BeginAlgorithm}{EndAlgorithm}
\algblock[BLOCK]{BeginBlock}{EndBlock}
\makeatother
 
% Нумерация блоков
 
\usepackage{caption}% http://ctan.org/pkg/caption
\captionsetup[ruled]{labelsep=period}
\renewcommand{\thealgorithm}{\arabic{algorithm}}
 

	
% Перевод данных об алгоритмах
\renewcommand{\listalgorithmname}{Список алгоритмов}
\floatname{algorithm}{Алг}

% Перевод команд псевдокода
\algrenewcommand\algorithmicwhile{\textbf{\underline{Цикл-пока}}}
\algrenewcommand\algorithmicdo{\textbf{\underline{}}}
\algrenewcommand\algorithmicrepeat{\textbf{\underline{Цикл}}}
\algrenewcommand\algorithmicuntil{\textbf{\underline{до}}}
\algrenewcommand\algorithmicend{\textbf{\underline{Конец}}}
\algrenewcommand\algorithmicif{\textbf{\underline{Если}}}
\algrenewcommand\algorithmicelse{\textbf{\underline{иначе}}}
\algrenewcommand\algorithmicthen{\textbf{\underline{то}}}
\algrenewcommand\algorithmicfor{\textbf{\underline{Цикл от}}}
\algrenewcommand\algorithmicforall{\textbf{Выполнить для всех}}
\algrenewcommand\algorithmicfunction{\textbf{\underline{Функция}}}
\algrenewcommand\algorithmicprocedure{\textbf{\underline{Процедура}}}
\algrenewcommand\algorithmicloop{\textbf{\underline{Зациклить}}}
\algrenewcommand\algorithmicrequire{\textbf{\underline{Условия:}}}
\algrenewcommand\algorithmicensure{\textbf{\underline{Обеспечивающие условия:}}}
\algrenewcommand\algorithmicreturn{\textbf{\underline{Возвратить}}}
\algrenewtext{EndWhile}{\textbf{\underline{кц}}}
\algrenewtext{EndLoop}{\textbf{\underline{кц}}}
\algrenewtext{EndFor}{\textbf{\underline{кц}}}
\algrenewtext{EndFunction}{\textbf{Конец функции}}
\algrenewtext{EndProcedure}{\textbf{Конец процедуры}}
\algrenewtext{EndIf}{\textbf{\underline{Всё}}}
\algrenewtext{BeginAlgorithm}{\textbf{\underline{Алг}}}
\algrenewtext{EndAlgorithm}{\textbf{\underline{Кон.}}}
\algrenewtext{BeginBlock}{\textbf{Начало блока. }}
\algrenewtext{EndBlock}{\textbf{Конец блока}}
\algrenewtext{ElsIf}{\textbf{\underline{иначе если }}}
\begin{document} % конец преамбулы, начало документа
\begin{titlepage}
	\begin{center}
		ФЕДЕРАЛЬНОЕ  ГОСУДАРСТВЕННОЕ АВТОНОМНОЕ \\
		ОБРАЗОВАТЕЛЬНОЕ УЧРЕЖДЕНИЕ ВЫСШЕГО ОБРАЗОВАНИЯ\\
		«НАЦИОНАЛЬНЫЙ ИССЛЕДОВАТЕЛЬСКИЙ УНИВЕРСИТЕТ\\
		«ВЫСШАЯ ШКОЛА ЭКОНОМИКИ»
	\end{center}
	
	\begin{center}
		\textbf{Московский институт электроники и математики}
	\end{center}
	\vspace{1ex}	
	\begin{center}
		Подчезерцев Алексей Евгеньевич, группа БИВ172
	\end{center}	
	\vspace{1ex}
	\begin{center}
		\textbf{КОМПЬЮТЕРНАЯ ИГРА «ЭРУДИТ» (SCRABBLE)}
	\end{center}	
	\vspace{2ex}
	\begin{center}
		Курсовая работа\\
		по направлению 09.03.01 Информатика и вычислительная техника\\
		студента образовательной программы бакалавриата\\
		«Информатика и вычислительная техника»
	\end{center}
	\vspace{2ex}
	\begin{flushright}
		Студент $\rule{5cm}{0.15mm}$ А.Е. Подчезерцев 
	\end{flushright}
	\vspace{3ex}
	\begin{flushright}
		Руководитель\\
		$\rule{5cm}{0.15mm}$ Е.А. Ерохина
	\end{flushright}
	\vfill
	\begin{center}
		Москва \the\year г.
	\end{center}
\end{titlepage}
\tableofcontents
\pagebreak
\section{Аннотация}
	Разрабатываем игрушку
	\pagebreak
\section{Словарь игры}
	\subsection{Условие задачи}
	Дан исходный массив слов и набор букв, которые уже есть на поле и у игрока. Необходимо выбрать из исходного списка слов такие, которые теоретически можно составить из данных букв
	\subsection{Постановка задачи}
	\underline{Дано:}
	
	$Words[0:n-1]$ - строки
	
	$Letters[0:m-1]$ - символьные
	
	\underline{Результат:}
	
	$NewWords[0:k-1]$ - строки
	
	\underline{При:}
	
	$n \ge 1, 1 \le m \le 32$
	
	\underline{Связь:}
	
	$k = n, NewWords[0:k-1] = Words[0:n-1]$, если $m = 32$ 
	
	$i=\overline{0, k-1}$
	
	$j=\overline{0, n-1}$
	
	$NewWords[i] = Words[j]$, если $\forall Words[j][e], \exists C, C \in Letters и Words[j][e] = C $
	
	\subsection{Внешняя спецификация}
	Данная функция не предусматривает взаимодействие программы с пользователем
	\pagebreak
	\subsection{Описание алгоритмов}	
	\begin{algorithm*}[!htp]
	\caption{Подготовка словаря}
	\begin{algorithmic}
		\State $k := 0$
		\For{$i:=0$ \textbf{\underline{до}} $n-1$}
			\State $flag := true$
			\State $j := 0$
			\While{$j<\text{длина}(Words[i])$ \textbf{\underline{и}} $flag$} 
				\If{$Words[i][j] \in Letters$}
					\State $j := j + 1$
				\Else
					\State $flag := false$
				\EndIf				
			\EndWhile
			\If{$flag$}
				\State $NewWords[k] := Words[i]$
				\State $k := k + 1$
			\EndIf
		\EndFor
	\end{algorithmic}
	\end{algorithm*}
	\subsection{Тесты}
	Тестовый словарь:
	
	биосфера блюз дворянство домолачивание заковывание изъян киноведение колеровщик координированность митраизм налавливание неминуемость одухотворенность окраина плавсостав поборник подхват приматывание пролысина сипловатость солододробилка топаз трином трехсотлетие умывание хранилище централизация шейх 

	Тесты обрабатывают тестовый словарь и проверяют длину итогового массива
	
	\begin{tabular}{|c|c|}
		\hline
		         Входные данные          & Контрольное значение \\ \hline
		АБВГДЕЖЗИЙКЛМНОПРСТУФХЦЧШЩЪЫЬЭЮЯ &    Длина словаря     \\ \hline
		               А                 &          0           \\ \hline
		              БЛЮЗ               &          1           \\ \hline
		АБВГДЕЖЗИЙКЛМНОПРСТУФХЦЧШЩЪЫЬЭЮ  &  Длина словаря - 3   \\ \hline
	\end{tabular}
	
\section{Использованные классы}
	\subsection{Matrix}
	За поиск новых слов в матрице отвечает класс Matrix. На вход подается два массива: один с координатами новых точек, другой с новыми буквами.
\end{document} % конец документа

